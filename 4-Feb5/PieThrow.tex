\documentclass{tufte-handout}

\title{More on Mathematical Induction}

\author{David Dobor}

\date{February 5, 2016} % without \date command, current date is supplied

%\geometry{showframe} % display margins for debugging page layout

\usepackage{graphicx} % allow embedded images
  \setkeys{Gin}{width=\linewidth,totalheight=\textheight,keepaspectratio}
  \graphicspath{{graphics/}} % set of paths to search for images
\usepackage{amsmath}  % extended mathematics
\usepackage{booktabs} % book-quality tables
\usepackage{units}    % non-stacked fractions and better unit spacing
\usepackage{multicol} % multiple column layout facilities
\usepackage{lipsum}   % filler text
\usepackage{fancyvrb} % extended verbatim environments
  \fvset{fontsize=\normalsize}% default font size for fancy-verbatim environments

% Standardize command font styles and environments
\newcommand{\doccmd}[1]{\texttt{\textbackslash#1}}% command name -- adds backslash automatically
\newcommand{\docopt}[1]{\ensuremath{\langle}\textrm{\textit{#1}}\ensuremath{\rangle}}% optional command argument
\newcommand{\docarg}[1]{\textrm{\textit{#1}}}% (required) command argument
\newcommand{\docenv}[1]{\textsf{#1}}% environment name
\newcommand{\docpkg}[1]{\texttt{#1}}% package name
\newcommand{\doccls}[1]{\texttt{#1}}% document class name
\newcommand{\docclsopt}[1]{\texttt{#1}}% document class option name
\newenvironment{docspec}{\begin{quote}\noindent}{\end{quote}}% command specification environment


\usepackage{listings}
\usepackage{color}

\definecolor{dkgreen}{rgb}{0,0.6,0}
\definecolor{gray}{rgb}{0.5,0.5,0.5}
\definecolor{mauve}{rgb}{0.58,0,0.82}

\lstset{frame=tb,
  language=matlab,
  aboveskip=3mm,
  belowskip=3mm,
  showstringspaces=false,
  columns=flexible,
  basicstyle={\small\ttfamily},
  numbers=none,
  numberstyle=\tiny\color{gray},
  keywordstyle=\color{blue},
  commentstyle=\color{dkgreen},
  stringstyle=\color{mauve},
  breaklines=true,
  breakatwhitespace=true,
  tabsize=3
}



\begin{document}

\maketitle% this prints the handout title, author, and date

\begin{abstract}
\noindent We have seen several examples of proofs using mathematical induction. Here we show another example illustrating a bit more creative use of this proof technique. \thanks{Adapted from Carmony, ``Odd Pie Fights'', \textit{Mathematics Teacher}, 1979}

\end{abstract}

\bigskip
\section{Odd Pie Fights}
\newthought{An odd number} of comp-sci students stand scattered around a yard at mutually distinct distances, each armed with a freshly baked apple pie. A whistle is suddenly blown and the pie throwing contest begins. Each contestant throws a pie at the contestant nearest to her. 

\bigskip
Use mathematical induction to show that there is at least one survivor. 

\bigskip
\textit{\underline{Proof:} } We will show that the following proposition $P(n)$ is true for all positive $n$:
\let\thefootnote\relax\footnotetext{Note that as $n$ runs through all positive integers, $2n + 1$ runs through all odd integers starting at 3 and greater. This is what we want -  we don't want to allow for a single pie-fighter; we need at least three - since we suspect that no sane person would want to engage into a pie fight with self. These are computer science  students, after all.}
 \begin{equation*}
 \begin{aligned}
P(n) = \{ \text{ There is a survivor whenever } 2n + 1  \text{ people standing at} \\ 
                                                       \text{mutually distinct distances throw pies at each other }\}
\end{aligned}
\end{equation*}


\newthought{\underline{Base Case:} $P(1)$ is true.} If $n = 1$, then we have $2n + 1 = 3$ people in the pie-fight. Let the three people be $A, B,$ and $C$. Let, without loss of generality, $A$ and $B$ be the closest pair.  Since the distances between all pairs  are different, we know that $C$ is closer to either $A$ or $B$. Let, again without loss of generality, $A$ be the closest guy to $C$. 

  What happens as the contest begins? $A$ and $B$ through pies at each other, and $C$ throws the pie at $A$. Thus we have the survivor, $C$. This proves the base case.
  
      

\newthought{\underline{Inductive Case:} If $P(k)$ is true, then $P(k+1)$ is true.} Assume that $P(k)$ is true for some odd integer $k > 3$. That is, we have at least one survivor when $2k + 1$ people pie-fight as described above. We must show that $P(k+1)$ is true. That is: we must show that there is at least one survivor whenever $2(k+1) + 1 = 2k + 3$ people pie-fight as described above.  

\bigskip
So suppose we have these $2k + 3$ people in the yard. Let $A$ and $B$ be the closest pair. When each person throws a pie, $A$ and $B$ throw pies at each other. 

\bigskip Now consider two cases: 1) Someone else throws a pie at either $A$ or $B$ and 2) No one else throws a pie  at either $A$ or $B$.
\newthought{\textit{Case 1: Someone else throws a pie at either $A$ or $B$.}} In this case \textit{at least } three pies are thrown at $A$ and $B$: 2 pies that they throw at each other and at least another pie that comes their way from somebody else. But this means that \textit{at most } $(2k+3) - 3 = 2k$ pies are thrown at the remaining $2k + 1$ people. But this guarantees that there is at least one survivor: there are fewer pies than people to be hit. 

\newthought{\textit{Case 2:  No one else throws a pie  at either $A$ or $B$.}} If nobody else throws a pie at either $A$ or $B$, then the proof is simpler. Consider removing $A$ and $B$ with their 2 pies from the $2k + 3$ people in the contest. We would be left with $2k + 1$ pie-fighting each other only. By the inductive hypothesis, there is at least one survivor amongst these $2k+1$ people. This person is also the survivor amongst the initial $2k + 3$ contestants (since $A$ and $B$ trow pies at each other).

\bigskip
This completes the proof. We proved the base case and we proved the inductive step using a proof by cases. From the principle of mathematical induction it follows that $P(n)$ is true for all positive integers $n$. 

\bigskip
Notice that the conclusion we reached here would be false if the contest set-up were the same as before but we had an even number of people in the yard. Can you see why is it possible for everyone to be hit with a pie in this case?
%\begin{marginfigure}
%  \includegraphics{graph}
%  \caption{A $O(n^2)$ algoritm scales horribly.}
%\end{marginfigure}



\end{document}


There are plenty of opportunities for making errors when using mathematical induction. Here is an example: we ``prove'' that all horses are of the same color.   

















